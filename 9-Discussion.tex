\selectlanguage{french}
\Chapter{DISCUSSION GÉNÉRALE}\label{sec:Discussion}

\section{Synthèse des travaux}

Les objectifs poursuivis dans le cadre de cette thèse se déclinaient comme suit : 

\begin{enumerate}
	\item Concevoir un élément chauffant nanocomposite conducteur en tenant compte des propriétés électriques et thermiques nécessaires au procédé de soudage par résistance. 
	\item Développer un procédé de soudage par résistance avec un élément chauffant nanocomposite entre deux adhérents en composite à matrice thermoplastique. 
	\item Établir une fenêtre d'opération pour le soudage par résistance avec un élément chauffant nanocomposite entre deux adhérents en composite à matrice thermoplastique. 
	\item Développer un procédé de soudage par résistance avec un élément chauffant nanocomposite entre un adhérent en composite à matrice thermoplastique et un adhérent en élastomère thermoplastique. 
\end{enumerate}

En premier lieu, un nanocomposite conducteur a été conçu et caractérisé. 
Pour ce faire, plusieurs compositions de nanocomposite ont été produites et évaluées. 
La composition démontrant les meilleures caractéristiques a été produite en plus grande quantité et utilisée pour démontrer la possibilité de l'utiliser comme élément chauffant. 
La preuve de concept du procédé de soudage utilisant ce nanocomposite a permis de valider le concept d'un élément chauffant nanocomposite. 

En second lieu, un modèle par éléments finis a été développé afin d'établir les limites de la fenêtre d'opération du procédé de soudage. 
Des essais de soudage supplémentaires en laboratoire sont venus confirmer la fenêtre identifiée. 
Grâce a un meilleur contrôle du procédé, il a été possible d'améliorer les performances mécaniques des joints. 

Finalement, la production de jonctions soudées flexibles à l'aide du procédé de soudage résistif a été explorée et quelques avancées ont été obtenues. 
Cette exploration a mené à la sélection du ULTEM STM1500 qui est compatible avec le PEI. 
Par contre, les travaux de cette thèse n'ont cependant pas permis de trouver des paramètres opérationnels permettant d'obtenir un soudage multi matériau rencontrant les requis techniques d'ArianeGroup. 
Certaines pistes supplémentaires ont été identifiées telles que le taux d'humidité dans le nanocomposite et l'élastomère ainsi que la possibilité de texturer le nanocomposite. 

Les objectifs 1 à 3 ont été couverts et atteints par les articles présentés aux chapitres \ref{sec:Theme1} et \ref{sec:Theme2}. 
Le premier article \cite{Brassard2019a} a été publié en février 2019 dans la revue \textit{Composites Part B: Engineering} après révision par les pairs. 
Cet article a déjà été reconnu et cité par d'autres chercheurs travaillant au développement de nouveaux éléments chauffants pour le soudage des composites à matrice thermoplastique \cite{Russello2019}. 
Le second article a été soumis au journal \textit{Composites Part B: Engineering} en octobre 2019 et est en attente de révision par les pairs. 
En ce qui concerne l'objectifs 4, ce dernier fait l'objet du chapitre \ref{sec:Theme3}. 