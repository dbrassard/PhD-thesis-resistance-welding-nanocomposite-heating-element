% Remerciements / Acknowledgements
%
%  Grâce aux remerciements, l'auteur attire l'attention du lecteur
% sur l'aide que certaines personnes lui ont apportée, sur leurs
% conseils ou sur toute autre forme de contribution lors de la
% réalisation de son mémoire. Le cas échéant, c'est dans cette section
% que le candidat doit témoigner sa reconnaissance à son directeur de
% recherche, aux organismes dispensateurs de subventions ou aux
% entreprises qui lui ont accordé des bourses ou des fonds de
% recherche.
\ifthenelse{\equal{\Langue}{english}}{
	\chapter*{ACKNOWLEDGEMENTS}\thispagestyle{headings}
	\addcontentsline{toc}{compteur}{ACKNOWLEDGEMENTS}
}{
	\chapter*{REMERCIEMENTS}\thispagestyle{headings}
	\addcontentsline{toc}{compteur}{REMERCIEMENTS}
}
%
Texte / Text.

%
%Ma femme Marie-Pier ainsi que mes enfants
%Jason R. Tavares
%Martine Dubé
%
%Nick Virgilio
%Marie-Claude Heuzey
%
%Wendell, Hamed, Faezeh, Simon, Alessio, Christina, Mélanie
%Vincent Rohart, Laurent Cormier, Louis-Charles Forcier
%David Vidal, Bruno Blais
%Daniel Pilon, Anic Deforges, Martine, Gino, Matthieu Gauthier, Robert Delisle, CLaire Cerclé
%Serge Plamondon
%
%Financement
%ArianeGroup et bourse M Neil R. Mitchell et Mme Danièle Dumais et la Fondation Polytechnique 


