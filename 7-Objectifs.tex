\selectlanguage{french}
\Chapter{OBJECTIFS DE LA RECHERCHE}\label{sec:Objectifs}

\section{Analyse des trous dans les connaissances actuelles}

Le procédé de soudures par résistance des composites thermoplastiques avec un élément chauffant en acier inoxydable est un procédé simple qui offre de bonnes propriétés mécaniques et qui peut facilement être mis à l'échelle. 
Sa faiblesse réside dans le manque d'interaction entre le polymère et l'élément chauffant. 
Celle-ci peut être contournée de deux façons \begin{inparaenum}[a)]
	\item en effectuant des traitements de surface permettant d'augmenter l'adhésion entre la matrice et l'acier inoxydable ou 
	\item en changeant la nature de l'élément chauffant. 
\end{inparaenum}
En ce qui concerne la première solution, des travaux de recherche sont en cours à l'École de technologie supérieure afin d'explorer cette option. 
Pour la seconde solution, un élément chauffant constitué d'un nanocomposite conducteur pourrait être employé. 
Le polymère de la matrice de ce nanocomposite devra pouvoir diffuser au travers du polymère des adhérents et l'ajout de particules conductrices le rendra conducteur. 
Ainsi, le nanocomposite développé servirait à la fois d'élément chauffant et offrira une zone riche en polymère pour produire une soudure. 
Cette seconde solution est inexplorée dans les articles précédemment publiés. 

Un autre trou existe dans les écrits recensés en rapport à la possibilité de souder un élastomère thermoplastique avec un autre polymère rigide. 
Dans le meilleur des cas, un mélange biphasique avec un élastomère et un autre polymère est utilisé dans une jonction, mais sans obtenir de diffusion entre les phases. 
Aucun cas documenté de soudage mixte entre un élastomère et un autre polymère n'a été trouvé. 

Ainsi, nous pouvons détecter deux besoins de recherche. 
Tout d'abord en ce qui concerne la production d'un élément chauffant nanocomposite permettant la soudure par résistance de composites thermoplastiques. 
En second lieu, par l'obtention d'une soudure mixte avec un élastomère. 

\section{Objectifs}

En considérant l'ensemble des connaissances publiées et des besoins du partenaire industriel, voici la liste des objectifs sur lesquels se concentreront ces travaux. 

\begin{enumerate}
	\item Développer un nanocomposite conducteur pouvant servir d'élément chauffant dans un procédé de soudage par résistance. 
	\item Produire une soudure par résistance entre des adhérents en composites thermoplastiques. 
	\item Établir une méthode de soudage par résistance pour la production d'une jonction flexible intégrant un élastomère thermoplastique entre des adhérents en composites thermoplastiques. 
\end{enumerate}

\section{Contenu des chapitres \ref{sec:Theme1} à \ref{sec:Theme3}}

Le contenu des prochains chapitres est une application directe des objectifs qui viennent d'être établis. 
En tout premier lieu, le chapitre \ref{sec:Theme1} présentera un article traitant du développement du nanocomposite ainsi que la preuve de concept nécessaire pour démontrer la possibilité de l'utiliser comme élément chauffant lors d'un soudage par résistance. 
En second lieu, le chapitre \ref{sec:Theme2} présente un article traitant de l'amélioration des performances du procédé par l'emploi d'un modèle numérique par éléments finis afin de cibler une fenêtre d'opération plus précise. 
En dernier lieu, le chapitre \ref{sec:Theme3} présente les travaux ayant été menés afin de développer une jonction flexible intégrant un élastomère thermoplastique. 