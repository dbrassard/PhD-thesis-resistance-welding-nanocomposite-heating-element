\selectlanguage{french}
\Chapter{CONTRIBUTIONS ET CONCLUSION}\label{sec:Contributions}

\section{Synthèse des travaux}

Les objectifs poursuivis dans le cadre de cette thèse se déclinaient comme suit : 

\begin{enumerate}
	\item Développer un nanocomposite conducteur pouvant servir d'élément chauffant dans un procédé de soudage par résistance. 
	\item Produire une soudure par résistance entre des adhérents en composites thermoplastiques. 
	\item Établir une méthode de soudage par résistance pour la production d'une jonction flexible intégrant un élastomère thermoplastique entre des adhérents en composites thermoplastiques. 
\end{enumerate}

Les travaux de cette thèse atteignent en partie ces objectifs. 
En premier lieu, un nanocomposite conducteur a été développé et caractérisé. 
Pour ce faire, plusieurs compositions de nanocomposite ont été produites et évaluées. 
La composition démontrant les meilleures caractéristiques a été produite en plus grande quantité et utilisée pour démontrer la possibilité de l'utiliser comme élément chauffant. 
En second lieu, le processus de soudage a été modélisé à l'aide d'un modèle par éléments finis afin d'établir une fenêtre de procédé. 
Des essais de soudage supplémentaires sont venus renforcer notre contrôle du procédé et ont permis de valider la fenêtre proposée. 
Finalement, la production de jonctions soudées flexibles à l'aide du procédé de soudage résistif a été explorée. 
Cette exploration a mené à la sélection du ULTEM STM1500 qui est compatible avec le PEI. 
Ces travaux n'ont cependant pas permis de déterminer des paramètres permettant d'obtenir un soudage multi matériau. 

Les objectifs 1 et 2 ont été couverts et atteints par les articles présentés aux chapitres \ref{sec:Theme1} et \ref{sec:Theme2}. 
Le premier article \cite{Brassard2019a} a été publié en février 2019 dans la revue \textit{Composites Part B: Engineering} après révision par les pairs. 
Cet article a déjà été reconnu et cité par d'autres chercheurs travaillant au développement de nouveaux éléments chauffants pour le soudage des composites à matrice thermoplastique \cite{Russello2019}. 
Le second article a été soumis au journal \textit{Composites Part B: Engineering} en octobre 2019 et est en attente de révision par les pairs après avoir été accepté par l'éditeur. 
En ce qui concerne l'objectif 3, ce dernier fait l'objet du chapitre \ref{sec:Theme3}. 
Certains travaux toujours en cours pourraient venir permettre de le compléter afin de préparer un troisième article.

\section{Contributions originales}

La recherche de nouveaux éléments chauffants a permis au soudage résistif de passer initialement d'éléments chauffants en fibre de carbone vers des éléments chauffants en acier inoxydable. 
Il est naturel dans ce domaine de poursuivre les recherches de nouvelles solutions \cite{Russello2019}. 
Dans sa forme actuelle, l'élément chauffant présenté dans le cadre de cette thèse n'est pas un substitut direct pour un grillage en acier inoxydable. 
Il présente des limitations.
Cependant il ouvre la porte vers une nouvelle classe d'éléments chauffants qui s'intègrent totalement dans la pièce obtenue. 
Ces travaux pourront servir de base à des recherches visant à raffiner cette première itération. 
De plus, comme pour les matériaux composites, les éléments chauffants nanocomposites peuvent être adaptés à l'application spécifique où ils seront employés. 
Un potentiel encore inexploré s'offre à eux. 

Dans un second temps, même si le soudage multi matériaux existe dans le cas de composites thermoplastiques et de composites thermodurcissables \cite{FernandezVillegas2015,Lionetto2018a} ou encore de composites thermoplastiques et de métaux \cite{Weidmann2018,Kruger2004,Balle2009,Goushegir2016}, aucun cas documenté n'existe pour le soudage d'une jonction flexible. 
Ce type de soudure n'a tout simplement jamais été réalisé. 
La preuve de la nécessité de ce type de jonction provient de la justification initiale de ce doctorat.
Notre partenaire, ArianeGroup, recherche un type de jonction qui n'existe tout simplement pas encore sur le marché et ces travaux de recherche ont eu pour but de répondre à ce besoin. 
Bien qu'aucune soudure rencontrant les requis du cahier des charges n'ait été produite, poser les bases afin de répondre éventuellement à ce besoin constitue déjà, en elle seule, une contribution originale. 

%%
%%  LIMITATIONS
%%
\section{Limitations et pistes d'amélioration}

Les solutions développées dans le cadre de ce doctorat ne sont pas sans failles. 
Bien identifier leurs limites permet de cerner des pistes d'amélioration et de formuler des projets de recherche subséquents. 

Une grande limitation du soudage à l'aide d'un élément chauffant nanocomposite, dans sa forme actuelle, réside dans son incapacité à produire des soudures avec des laminés autres qu'unidirectionnels. 
La plus faible conductivité électrique du nanocomposite par rapport aux éléments chauffants en acier inoxydable les rend plus susceptibles aux fuites de courant au travers du laminé quand des fibres de carbone sont alignées dans le même sens que le courant. 
Lors du soudage, les voltages appliqués sont approximativement douze fois plus élevés lorsqu'un élément chauffant nanocomposite est employé. 
En optimisant la nature du nanocomposite ou en augmentant le nombre de nanotubes de carbone, il serait possible d'obtenir des conductivités plus élevées qui réduiraient ce phénomène. 

Cependant, une balance doit être conservée en ce qui concerne la composition du nanocomposite, une grande quantité de nanotubes dans le nanocomposite limite la mobilité des chaînes de polymère. 
Ceci entraîne des difficultés lors de la mise en forme du nanocomposite, mais plus particulièrement lors du soudage.
En raison de la faible mobilité des chaînes, il est nécessaire de chauffer le joint à des températures plus élevées. 
Ce niveau de chauffe cause des problèmes de dégradation thermique du nanocomposite. 
Même pour les soudures réussies, on peut remarquer un début de dégradation thermique lors de l'observation des faciès de rupture. 
La faible mobilité des chaînes de polymère en raison de la quantité de nanotubes peut également expliquer une partie des problèmes rencontrée lors des essais de soudage multi matériaux. 
Ce facteur pousse vers des nanocomposites avec une fraction massique plus faible de nanotubes de carbone. 

Une autre limite en ce qui a trait au soudage réside dans la faible résilience mécanique du nanocomposite. 
En raison de la quantité de nanotubes, le nanocomposite présente un comportement purement élastique sans plateau plastique lors d'essais de traction. 
Des ruptures fragiles ont pu être observées non seulement lors des essais mécaniques avec le nanocomposite, mais également lors de l'analyse des faciès de rupture des joints soudés. 
Une augmentation de la résilience du nanocomposite pourrait également être obtenue en réduisant le chargement en nanotubes de carbone. 

Comme on peut le voir, la nature du nanocomposite sera le sujet que devraient traiter les prochaines recherches à propos du soudage avec un élément chauffant nanocomposite. 
La composition actuelle a permis de démontrer le procédé, mais considérer uniquement la conductivité électrique lors du choix de composition n'est pas suffisant pour obtenir un bon élément chauffant.

\section{Recherches futures}

Afin de répondre aux défis mentionnés dans la précédente section, plusieurs pistes de recherche sont possibles.  
La conductivité électrique, la résilience et la mobilité des chaînes de polymères devront être considérées pour la prochaine génération d'élément chauffant. 
Les améliorations requises nécessiteront des travaux sur plusieurs plans : 

\begin{itemize}
	\item Développer un traitement de fonctionnalisation des nanotubes qui permettra de réduire la résistance de contact et ainsi obtenir une plus grande conductivité avec un plus petit nombre de nanotubes. 
	\item Quantifier l'effet de l'ajout de nanoparticules sur la ductilité des nanocomposites et trouver une composition permettant d'équilibrer les requis de conductivité électrique avec les requis de résilience. 
	\item Déterminer plus précisément l'effet des charges sur le temps de reptation des chaînes de polymères et l'impact que cela entraîne sur les conditions de soudure. Cet effet devra également être pris en compte pour établir la composition de l'élément chauffant. 
	\item Afin de limiter les contraintes thermiques imposées au nanocomposite, il faudrait explorer la possibilité de surmouler les éléments chauffants ou de les intégrer durant la fabrication des composants. Il est possible qu'il soit nécessaire d'insérer des couches isolantes entre le composite et l'élément chauffant. 
	\item Après avoir modifié la composition de l'élément chauffant, il faudra évaluer l'effet  des modifications sur la fenêtre de procédé. Augmenter la mobilité des chaînes pourrait permettre d'élargir la fenêtre d'opération et éviter d'opérer à proximité des conditions menant à la dégradation du nanocomposite. 
	\item Dans le contexte du soudage multi matériaux, les soudages obtenus ont été produits avec des puissances relativement élevées et des temps de soudages très courts. Peut-être serait-il également approprié d'explorer des temps de soudage plus long avec des puissances plus faibles. 
	\item Bien entendu, un nanocomposite conducteur n'est peut-être pas la solution finale dans cette exploration pour découvrir de nouveaux éléments chauffants. La découverte de nouvelles alternatives doit être encouragée. 
	\item Finalement, une soudure n'est pas l'unique façon d'obtenir une jonction entre des composants. L'utilisation d'une surface texturée augmentant la rugosité peut permettre d'obtenir un blocage mécanique entre les phases pour certains types de chargement. Ces joints n'atteindront pas les performances de jonctions soudées, mais ils peuvent rencontrer les requis pour certaines applications. 
\end{itemize}

Le développement initial d'un élément chauffant nanocomposite s'inscrit dans le continuum de la recherche pour le soudage résistif des matériaux composites. 
Ce projet n'est pas une finalité, mais un tremplin vers le développement de nouvelles solutions. 