\selectlanguage{french}
\Chapter{CONCLUSION ET RECOMMANDATIONS}\label{sec:Contributions}

\section{Contributions originales}

La recherche de nouveaux éléments chauffants a permis au soudage résistif de passer initialement d'éléments chauffants en fibre de carbone vers des éléments chauffants en acier inoxydable. 
Il est naturel dans ce domaine de poursuivre les recherches de nouvelles solutions \cite{Russello2019}. 
Dans sa forme actuelle, l'élément chauffant présenté dans le cadre de cette thèse n'est pas un substitut direct pour un grillage en acier inoxydable, mais plutôt un complément. 
Il présente des limitations tout en présentant des avantages distincts.
Il ouvre la porte vers une nouvelle classe d'éléments chauffants qui s'intègrent totalement dans la pièce obtenue. 
Ces travaux pourront servir de base à des recherches visant à raffiner cette première itération. 
De plus, comme pour les matériaux composites, les éléments chauffants nanocomposites peuvent être adaptés à l'application spécifique où ils seront employés. 
Il pourraient, par exemple, être intégrés à des composants afin de servir d'éléments chauffants pour le dégivrage. 
Également, leur conductivité électrique pourrait être utilisée dans les systèmes de protection contre la foudre des aéronefs en composite en assurant une continuité électrique entre les composants assemblés. 
Un potentiel encore inexploré s'offre à eux. 

Dans un second temps, dans le cadre de la revue de la littérature pour le soudage par résistance, le contrôle de la source afin de l'opérer avec une puissance constante n'a jamais été présenté auparavant.  
Ce mode d'opération permet un meilleur contrôle de la puissance dissipée dans le joint et permet d'éliminer l'effet de la variation de résistance de l'élément chauffant que l'on rencontre même chez les éléments chauffants en acier inoxydable. 
Même si ce mode d'opération devenait nécessaire pour les éléments chauffants nanocomposites, il a permis d'améliorer le contrôle du procédé de soudage lors de la production de joints avec un élément chauffant en acier inoxydable selon les observations d'un collègue travaillant avec ce type d'élément chauffant. 

Finalement, même si le soudage multi matériaux existe dans le cas de liaisons entre des composites thermoplastiques et des composites thermodurcissables \cite{FernandezVillegas2015,Lionetto2018a} ou encore pour des liaisons entre des composites thermoplastiques et des métaux \cite{Weidmann2018,Kruger2004,Balle2009,Goushegir2016}, aucun cas documenté n'existe pour le soudage d'une jonction flexible. 
La preuve de la nécessité de ce type de jonction provient de la justification initiale de ce doctorat.
Notre partenaire, ArianeGroup, recherche un type de jonction qui n'existe tout simplement pas encore sur le marché et ces travaux de recherche ont eu pour but de répondre à ce besoin. 
Les adhésifs peuvent remplir une partie de leur besoins à court terme, mais les problèmes de sensibilité aux contaminants et le manque de robustesse de ces jonctions force le développement de nouvelles solutions. 
Bien qu'aucune soudure rencontrant les requis du cahier des charges n'ait été produite, pour le moment, poser les bases afin de répondre éventuellement à ce besoin constitue déjà une contribution originale. 
De plus, les travaux présentés on permis de sélectionner le copolymère multi bloc ULTEM STM1500 comme candidat potentiel pour obtenir des soudures. 

\section{Recommandations}

Les solutions développées dans le cadre de ce doctorat ne sont pas sans failles. 
Bien identifier leurs limites permet de cerner des pistes d'amélioration et de formuler des projets de recherche subséquents. 

Une grande limitation du soudage à l'aide d'un élément chauffant nanocomposite, dans sa forme actuelle, réside dans son incapacité à produire des soudures avec des laminés autres qu'unidirectionnels. 
La plus faible conductivité électrique du nanocomposite par rapport aux éléments chauffants en acier inoxydable les rend plus susceptibles aux fuites de courant au travers du laminé quand des fibres de carbone sont alignées dans le même sens que le courant. 
Lors du soudage, les voltages appliqués sont approximativement douze fois plus élevés que lorsqu'un élément chauffant en acier inoxydable est employé. 
En optimisant la nature du nanocomposite ou en augmentant le nombre de nanotubes de carbone, il serait possible d'obtenir des conductivités plus élevées qui réduiraient ce phénomène. 
Afin de résoudre ce problème, quelques pistes de solutions peuvent être envisagées. 

\begin{itemize}
	\item Développer un traitement de fonctionnalisation des nanotubes qui permettra de réduire la résistance de contact et ainsi obtenir une plus grande conductivité.  
	\item L'ajout de couches de PEI vierge de chaque côté du nanocomposite, moulées avec l'élément chauffant ou en tant que films intercalés, agissant comme isolant pourrait également aider à réduire les risques de fuite de courant. 
\end{itemize}

Cependant, une balance doit être conservée en ce qui concerne la composition du nanocomposite, une grande quantité de nanotubes dans le nanocomposite limite la mobilité des chaînes de polymère. 
Ceci entraîne des difficultés lors de la mise en forme du nanocomposite, mais plus particulièrement lors du soudage.
En raison de la faible mobilité des chaînes, il est nécessaire de chauffer le joint à des températures plus élevées. 
Ce niveau de chauffe cause des problèmes de dégradation thermique du nanocomposite. 
%Même pour les soudures réussies, on peut remarquer un début de dégradation thermique lors de l'observation des faciès de rupture. 
La faible mobilité des chaînes de polymère en raison de la quantité de nanotubes peut également expliquer une partie des problèmes rencontrée lors des essais de soudage multi matériaux. 
Ce facteur pousse vers des nanocomposites avec une fraction massique plus faible de nanotubes de carbone. 

\begin{itemize}
	\item Des traitements de fonctionnalisation des nanotubes pourraient permettre non seulement d'augmenter la conductivité du composite mais également d'utiliser un plus petit nombre de nanotubes afin de limiter leur effet sur a mobilité des chaînes de polymère. 
	\item Il faudra déterminer plus précisément l'effet des charges sur le temps de reptation des chaînes de polymères et l'impact que cela entraîne sur les conditions de soudure. 
	Cet effet devra être pris en compte lors de l'optimisation de la composition de l'élément chauffant. 
	\item Afin de limiter les contraintes thermiques imposées au nanocomposite, il faudrait explorer la possibilité de surmouler les éléments chauffants ou de les intégrer durant la fabrication des composants. 
	Il est possible qu'il soit nécessaire d'insérer des couches isolantes entre le composite et l'élément chauffant. 
	En plus d'agir comme isolant, si ces couches sont des films de PEI vierge, ils peuvent fournir, à l'interface du joint soudé, un milieu exempt de nanocharges avec une plus grande mobilité des chaînes de polymère que dans le nanocomposite lui-même. 
	\item Après avoir modifié la composition de l'élément chauffant, il faudra évaluer l'effet  des modifications sur la fenêtre de procédé. 
	Augmenter la mobilité des chaînes pourrait permettre d'élargir la fenêtre d'opération et éviter d'opérer à proximité des conditions menant à la dégradation du nanocomposite. 
\end{itemize}

Une autre limite en ce qui a trait au soudage réside dans la faible résilience mécanique du nanocomposite. 
En raison de la quantité de nanotubes, le nanocomposite présente un comportement purement élastique sans plateau plastique lors d'essais de traction. 
Des ruptures fragiles ont pu être observées non seulement lors des essais mécaniques avec le nanocomposite, mais également lors de l'analyse des faciès de rupture des joints soudés. 
Une augmentation de la résilience du nanocomposite pourrait également être obtenue en réduisant le chargement en nanotubes de carbone. 

\begin{itemize}
	\item Il faudra quantifier l'effet de l'ajout de nanoparticules sur la ductilité des nanocomposites et trouver une composition permettant un compromis entre les requis de conductivité électrique et la résilience. 
\end{itemize}

Comme on peut le voir, la nature du nanocomposite sera le sujet dont devraient traiter les prochaines recherches à propos du soudage avec un élément chauffant nanocomposite. 
La composition actuelle a permis de démontrer le procédé, mais considérer uniquement la conductivité électrique lors du choix de composition n'est pas suffisant pour obtenir un bon élément chauffant.

En ce qui concerne le soudage multi matériaux, les jonctions obtenues n'ont pas rencontrés les requis du cahier des charges. 
Il y a cependant quelques pistes de recherches qui nécessitent d'être poussées plus loin. 

\begin{itemize}
	\item Les conditions de soudage évaluées utilisaient des puissances relativement élevées et des temps de soudages très courts. 
	Il serait également approprié d'explorer des temps de soudage plus long avec des puissances plus faibles afin de réduire les contraintes thermiques sur l'élastomère. 
	\item Il faudrait sécher les adhérents et éléments chauffants pour éliminer l'effet de l'humidité absorbée sur l'issue des résultats. 
	Il est déjà documenté que la présence d'humidité dissoute dans un composite peut mener à l'apparition de porosités dans un joint soudé par résistance. 
	\item Finalement, une soudure n'est pas l'unique façon d'obtenir une jonction entre des composants. 
	L'utilisation d'une surface texturée augmentant la rugosité peut permettre d'obtenir un blocage mécanique entre les phases pour certains types de chargement. 
	Ces joints n'atteindront pas les performances de jonctions soudées, mais ils peuvent rencontrer les requis pour certaines applications. 
\end{itemize}

Le développement initial d'un élément chauffant nanocomposite s'inscrit dans le continuum de la recherche pour le soudage résistif des matériaux composites. 
Ce projet n'est pas une finalité, mais un tremplin vers le développement de nouvelles solutions. 