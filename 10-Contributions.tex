\selectlanguage{french}
\Chapter{CONCLUSION ET RECOMMANDATIONS}\label{sec:Contributions}

\section{Contributions originales}

La recherche de nouveaux éléments chauffants a permis au soudage résistif de passer initialement d'éléments chauffants en fibres de carbone vers des éléments chauffants en acier inoxydable. 
Il est donc naturel dans ce domaine de poursuivre les recherches pour développer de nouvelles solutions. 
Dans sa forme actuelle, l'élément chauffant présenté dans le cadre de cette thèse n'est pas un substitut direct pour un grillage en acier inoxydable, mais plutôt un complément. 
Il présente des limitations tout en bénéficiant d'avantages distincts.
Au niveau des propriétés mécaniques, avec un nanocomposite contenant 10\% massique de nanotubes de carbone, les essais en laboratoire ont permis d'obtenir une résistance en cisaillement maximale moyenne de \SI{24.9}{\mega\pascal}. 
De plus, les résultats des essais de cisaillement, pour l'ensemble des points se situant dans la fenêtre d'opération, se retrouvent tous à l'intérieur de la plage d'intervalle de confiance à 95\%. 
Les variations visibles des moyennes présentées (Fig. \ref{fig:2_Fig7}) proviennent donc simplement de fluctuations normales et aucun ensemble de paramètres en particulier ne se détache du lot. 
Ces résultats restent également inférieurs aux résistances pouvant être obtenues avec un élément chauffant en acier inoxydable qui peuvent atteindre jusqu'à \SI{50}{\mega\pascal}. 
Toutefois, cette recherche ouvre la porte vers une nouvelle classe d'éléments chauffants qui s'intègrent totalement dans la pièce obtenue. 
Ces travaux pourront servir de base à des recherches visant à raffiner cette première itération. 
De plus, comme pour les matériaux composites, les éléments chauffants nanocomposites peuvent être adaptés à l'application spécifique où ils seront employés. 
Ils pourraient, par exemple, être intégrés à des composants afin de servir d'éléments chauffants pour le dégivrage. 
Également, leur conductivité électrique pourrait être utilisée dans les systèmes de protection contre la foudre des aéronefs en composite, et ce, en assurant une continuité électrique entre les composants assemblés. 
Les échantillons en nanocomposite présentent encore un énorme potentiel pour la recherche et son application. 

Dans le cadre de la revue de la littérature concernant le soudage par résistance, le contrôle de la source avec une puissance constante n'a jamais été présenté avant cette thèse.  
Ce mode d'opération permet un meilleur contrôle de la puissance dissipée dans le joint et permet d'éliminer l'effet de la variation de résistance de l'élément chauffant, que l'on rencontre même avec les éléments chauffants en acier inoxydable. 
Même si le développement de ce mode d'opération devenait nécessaire pour les éléments chauffants nanocomposites de ce projet, il a néanmoins permis d'améliorer le contrôle du procédé de soudage lors de la production de joints avec un élément chauffant en acier inoxydable, selon les observations d'un collègue travaillant avec ce type d'élément chauffant. 

Finalement, même si le soudage multimatériaux existe dans le cas de liaisons entre des composites thermoplastiques et des composites thermodurcissables \cite{FernandezVillegas2015,Lionetto2018a} ou encore pour des liaisons entre des composites thermoplastiques et des métaux \cite{Weidmann2018,Kruger2004,Balle2009,Goushegir2016}, aucun cas documenté n'existe pour le soudage d'une jonction flexible. 
La preuve de la nécessité de ce type de jonction provient de la problématique initiale de cette thèse.
Notre partenaire, ArianeGroup, recherche un type de jonction qui n'existe tout simplement pas encore sur le marché et ces travaux de recherche ont eu pour but de répondre à ce besoin. 
Les adhésifs peuvent remplir une partie de leurs requis à court terme, mais les problèmes de sensibilité aux contaminants et le manque de robustesse de ces jonctions force le développement de nouvelles solutions. 
Bien qu'aucune soudure rencontrant les requis du cahier des charges n'ait été produite pour le moment, poser les bases théoriques constitue déjà une contribution originale dans l'optique de développer éventuellement une solution appliquée à leurs besoins. 
De plus, les travaux présentés ont permis de sélectionner le copolymère multiblocs PEI-siloxane comme candidat potentiel pour obtenir des soudures. 

\section{Limites et recommandations}

Les solutions développées dans le cadre de cette thèse ne sont pas sans failles. 
Bien identifier leurs limites permet de cerner des pistes d'amélioration et de formuler des projets de recherche subséquents. 

Tout d'abord, au vu d'éléments complémentaires trouvés tout au long du projet, il a été établi que l'utilisation de la méthode FTIR présentée en annexe du premier article (voir Annexe \ref{sec:Annexe_B}) n'était pas la bonne façon de déterminer la présence ou l'absence de dégradation. 
Le spectre présenté (Fig. \ref{fig:FTIR_spectra}) est de piètre qualité et ne fournit aucune information utile quant à l'état de dégradation du nanocomposite. 
La figure est cependant conservée dans cette thèse, puisqu'elle fait partie des documents complémentaires publiés avec l'article. 
Afin de valider l'absence de dégradation du PEI qui entrainerait une réduction de ses propriétés mécaniques, il serait approprié de réaliser des mesures de la masse moléculaire du PEI après le procédé de soudage. 

Également, lors des essais de soudage, les conditions hygrothermiques des matériaux n'ont pas été considérées malgré l'importance qu'elles peuvent avoir. 
Les porosités observées aux interfaces ainsi que dans l'élastomère peuvent provenir de l'absence de contrôle de ce paramètre clé. 
Ce facteur aurait dû être pris en compte et devra certainement être considéré lors des développements futurs. 

Un élément ayant été intégré tardivement au projet est l'enrichissement des surfaces à souder avec des films de PEI. 
L'absence des films de PEI a fait en sorte que les températures nécessaires au soudage n'étaient plus dans la plage suggérée pour un procédé de soudage avec un film de polymère amorphe, en plus de nécessiter la fusion complète du PEEK. 
Les prochains essais de soudages devraient donc intégrer des films de PEI en surface afin de réduire les températures nécessaires.  

Une grande limitation du soudage à l'aide d'un élément chauffant nanocomposite, dans sa forme actuelle, réside dans son incapacité à produire des soudures avec des laminés autres qu'unidirectionnels. 
La plus faible conductivité électrique du nanocomposite par rapport à celle des éléments chauffants en acier inoxydable rend les éléments chauffants plus susceptibles aux fuites de courant au travers du laminé quand des fibres de carbone sont alignées dans le même sens que le courant. 
Lors du soudage, les voltages appliqués sont approximativement douze fois plus élevés que lorsqu'un élément chauffant en acier inoxydable est employé. 
En optimisant la nature du nanocomposite ou en augmentant le nombre de nanotubes de carbone, il serait possible d'obtenir des conductivités plus élevées qui réduiraient ce phénomène. 
Afin de résoudre ce problème, quelques pistes de solutions peuvent être envisagées. 

\begin{itemize}
	\item Développer un traitement de fonctionnalisation des nanotubes qui permettra de réduire la résistance de contact et ainsi obtenir une plus grande conductivité.  
	\item Considérer l'emploi de plusieurs types de nanoparticules pour tirer profit de l'effet de synergie et obtenir des conductivités électriques plus élevées. 
	\item Ajouter des couches de PEI vierge de chaque côté du nanocomposite, moulées avec l'élément chauffant ou en tant que films intercalés, afin d'agir comme isolant et d'aider à réduire les risques de fuites de courant. 
\end{itemize}

Cependant, un équilibre doit être conservé en ce qui concerne la composition du nanocomposite. 
Une grande quantité de nanotubes dans le nanocomposite limite la mobilité des chaines de polymères. 
Ceci entraine des difficultés lors de la mise en forme du nanocomposite, mais plus particulièrement lors du soudage.
En raison de la faible mobilité des chaines, il est nécessaire de chauffer le joint à des températures plus élevées. 
Ce niveau de chauffe cause des problèmes de dégradation thermique du nanocomposite. 
La faible mobilité des chaines de polymères en raison de la quantité de nanotubes peut également expliquer une partie des problèmes rencontrés lors des essais de soudage multimatériaux. 
Ce facteur pousse vers des nanocomposites avec une fraction massique plus faible de nanotubes de carbone. 
En ce sens: 

\begin{itemize}
	\item Des traitements de fonctionnalisation des nanotubes pourraient permettre non seulement d'augmenter la conductivité du composite, mais également d'utiliser un plus petit nombre de nanotubes afin de limiter leur effet sur la mobilité des chaines de polymères. 
	\item L'emploi de plusieurs types de nanoparticules pour tirer profit de l'effet de synergie afin de réduire les quantités totales de nanoparticules nécessaires pourrait être considéré. 
	\item Il faudrait déterminer plus précisément l'effet des charges sur le temps de reptation des chaines de polymères et l'impact que cela entraine sur les conditions de soudure. 
	Cet effet devrait être pris en compte lors de l'optimisation de la composition de l'élément chauffant. 
	\item Afin de limiter les contraintes thermiques imposées au nanocomposite, il faudrait explorer la possibilité de surmouler les éléments chauffants ou de les intégrer durant la fabrication des composants. 
	Il pourrait être nécessaire d'insérer des couches isolantes entre le composite et l'élément chauffant. 
	En plus d'agir comme isolant, si ces couches étaient des films de PEI vierge, elles pourraient fournir, à l'interface du joint soudé, un milieu exempt de nanocharges avec une plus grande mobilité des chaines de polymères que dans le nanocomposite lui-même. 
	\item Après avoir modifié la composition de l'élément chauffant, il faudrait évaluer l'effet  des modifications sur la fenêtre de procédé. 
	Augmenter la mobilité des chaines pourrait permettre d'élargir la fenêtre d'opération et éviter d'opérer à proximité des conditions menant à la dégradation du nanocomposite. 
\end{itemize}

Une autre limite en ce qui a trait au soudage réside dans la faible résilience mécanique du nanocomposite. 
En raison de la quantité de nanotubes, le nanocomposite présente un comportement purement élastique sans plateau plastique lors d'essais de traction. 
Des ruptures fragiles ont pu être observées non seulement lors des essais mécaniques avec le nanocomposite, mais également lors de l'analyse des faciès de rupture des joints soudés. 
Une augmentation de la résilience du nanocomposite pourrait être obtenue en réduisant le chargement en nanotubes de carbone. 
En ce sens: 

\begin{itemize}
	\item Il faudrait quantifier l'effet de l'ajout de nanoparticules sur la ductilité des nanocomposites et trouver une composition permettant un compromis entre les requis de conductivité électrique et la résilience. 
\end{itemize}

Un PEI avec une faible masse moléculaire a été sélectionné pour la fabrication du nanocomposite. 
Ce PEI avait l'avantage de réduire le temps nécessaire à la reptation des chaines, mais en offrant des propriétés mécaniques réduites. 
Afin d'augmenter la résistance mécanique du nanocomposite, il serait intéressant d'évaluer la possibilité d'utiliser du PEI à chaines longues ou, encore, un mélange de PEI avec diverses longueurs de chaines. 

Finalement, la conception initiale du nanocomposite avait comme seul critère la conductivité électrique. 
Cette cible peut être atteinte de plusieurs façons. 
La solution actuelle, atteignant la cible de conductivité électrique, présente fort probablement un haut niveau d'agglomération des nanotubes de carbone. 
Ces agglomérats empêchent de tirer pleinement profit des propriétés des nanotubes de carbone. 
Une prochaine boucle d'optimisation du nanocomposite devrait prendre en compte l'état de dispersion des charges conductrices. 

Comme on peut le constater, la nature du nanocomposite est un sujet dont devraient traiter les prochaines recherches à propos du soudage avec un élément chauffant nanocomposite. 
La composition actuelle a permis de démontrer le procédé, mais considérer uniquement la conductivité électrique lors du choix de composition n'est pas suffisant pour obtenir un bon élément chauffant.
Effectuer des boucles itératives quant à la nature et à la composition du nanocomposite permettrait d'optimiser les matériaux pour le procédé de soudage par résistance. 

En ce qui concerne le soudage multimatériaux, les jonctions obtenues n'ont pas rencontré les requis du cahier des charges. 
Il y a cependant quelques pistes de recherches qui nécessiteraient d'être poussées plus loin. 

\begin{itemize}
	\item Les conditions de soudage évaluées utilisaient des puissances relativement élevées et des temps de soudages très courts. 
	Il serait approprié d'explorer des temps de soudage plus long avec des puissances plus faibles afin de réduire les contraintes thermiques sur l'élastomère. 
	\item Il faudrait sécher les adhérents et les éléments chauffants pour éliminer l'effet de l'humidité absorbée. 
	Il est déjà documenté que la présence d'humidité dissoute dans un composite peut mener à l'apparition de porosités dans un joint soudé par résistance \cite{Shi2014}. 
	\item Finalement, une soudure n'est pas l'unique façon d'obtenir une jonction entre des composants. 
	L'utilisation d'une surface texturée augmentant la rugosité peut permettre d'obtenir un blocage mécanique entre les phases pour certains types de chargement. 
	Ces joints n'atteindront pas les performances de jonctions soudées, mais ils pourraient rencontrer les requis pour certaines applications. 
\end{itemize}

Le développement initial d'un élément chauffant nanocomposite s'inscrit dans le continuum de la recherche pour le soudage résistif des matériaux composites. 
Ce projet n'est pas une finalité, mais un tremplin vers le développement de nouvelles solutions. 