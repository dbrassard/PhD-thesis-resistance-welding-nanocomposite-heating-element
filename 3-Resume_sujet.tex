\selectlanguage{french}
% Résumé du mémoire.
%
\chapter*{RÉSUMÉ}\thispagestyle{headings}
\addcontentsline{toc}{compteur}{RÉSUMÉ}

Le soudage par résistance est une méthode rapide et efficace pour joindre des pièces en composites thermoplastiques. 
En collaboration avec le partenaire industriel, ArianeGroup, ce projet a pour but de développer un élément chauffant nanocomposite pouvant servir à produire une jonction flexible en soudant des pièces de composite thermoplastique et d'élastomère thermoplastique.
Le développement de ce joint flexible soudé a pour but de remplacer une jonction par adhésifs entre les réservoirs et les jupettes sur un lanceur de satellite. 
Le développement de ce joint a été réalisé en 4 étapes. 
Tout d'abord, il a été nécessaire de concevoir un nanocomposite conducteur aux propriétés adaptées pour servir d'élément chauffant durant le soudage par résistance. 
Par la suite, cet élément chauffant a été utilisé pour développer un procédé de soudage entre des adhérents en composite thermoplastique. 
Ensuite, une fenêtre d'opération a été établie afin d'améliorer notre contrôle du procédé ainsi que les performances des joints obtenus. 
Finalement, l'élément chauffant nanocomposite a été employé afin de développer un procédé de soudage avec un élastomère thermoplastique afin de former une jonction flexible. 

Afin d'obtenir les propriétés électriques requises pour le soudage et après avoir évalué plusieurs compositions, un nanocomposite composé d'une matrice de PEI et d'une fraction massique de 10\% de nanotubes de carbone a été sélectionné. 
En raison de variabilité dans les résultats lorsque le contrôle du procédé était effectué en employant un voltage constant, un mode de contrôle à puissance constante a été développé et appliqué avec succès pour produire des jonctions soudées avec l'élément chauffant nanocomposite.  

En utilisant les données recueillies durant les essais de soudage, un modèle par éléments finis a été développé afin de prévoir la distribution de température dans le joint et d'établir la fenêtre d'opération. 
Avec celle-ci, des essais complémentaires de soudage ont été réalisés et une résistance maximale de \SI[locale=FR]{24.9}{\mega\pascal} a été atteinte lors d'essais en simple cisaillement, soit une amélioration de 28\% par rapport aux résultats précédents. 
Les températures atteintes dans le joint lors des essais de soudage sont cependant plus élevées que les températures traditionnellement nécessaires avec un élément chauffant en acier inoxydable et entraînent de la dégradation thermique dans le joint. 
Il est postulé que la forte fraction de nanotubes de carbone limite la mobilité des chaînes de polymère et retarde ainsi le soudage. 
De plus, la forte proportion de nanotubes réduit grandement la ductilité ainsi que la résistance du nanocomposite. 

Les derniers travaux présentés se sont ensuite concentrés sur le développement d'un procédé de soudage pour produire une jonction flexible avec un élastomère thermoplastique. 
Après des essais infructueux avec des élastomères SIS et SEBS, des résultats encourageants ont été obtenus lors d'essais de soudage isotherme dans une étuve sous vide et une presse chauffante pour l'élastomère ULTEM STM1500. 
Malgré sa bonne stabilité thermique à température élevée, cet élastomère n'a pas réussi à produire une soudure rencontrant les cibles de performance d'ArianeGroup. 
Encore une fois, la température élevée du procédé et le manque de mobilité des chaînes pourraient être les causes de ces résultats négatifs. 

Finalement, en se basant sur les résultats obtenus, des pistes de recherche sont proposées afin d'améliorer le procédé de soudage par résistance avec un élément chauffant nanocomposite. 





%Le résumé est un bref exposé du sujet traité, des objectifs visés,
%des hypothèses émises, des méthodes expérimentales utilisées et de
%l'analyse des résultats obtenus. On y présente également les
%principales conclusions de la recherche ainsi que ses applications
%éventuelles. En général, un résumé ne dépasse pas quatre pages.
%
%Le résumé doit donner une idée exacte du contenu du mémoire ou de la thèse. Ce ne
%peut pas être une simple énumération des parties du document, car il
%doit faire ressortir l'originalité de la recherche, son aspect
%créatif et sa contribution au développement de la technologie ou à
%l'avancement des connaissances en génie et en sciences appliquées.
%Un résumé ne doit jamais comporter de références ou de figures.
