\tikzset{
	basic/.style  = {draw, text width=6cm, drop shadow, rectangle},
	root/.style   = {basic, rounded corners=2pt, thin, align=center,
                   fill=gray!30},
	level 2/.style = {basic, rounded corners=5.5pt, thin,align=center, fill=gray!10,
                   text width=8em},
	level 3/.style = {basic, thin, align=left, fill=white, text width=6.em}
}


\begin{tikzpicture}[
	level 1/.style={sibling distance=45mm},
	edge from parent/.style={->,draw},
	>=latex,
	scale=0.9,
	level distance=1.8cm]

% root of the the initial tree, level 1
\node[root] {Procédés de soudage}
% The first level, as children of the initial tree
	child {node[level 2] (c1) {Chauffage en volume}}
	child {node[level 2] (c2) {Chauffage par friction}}
	child {node[level 2] (c3) {Chauffage électromagnétique}}
	child {node[level 2] (c4) {Soudage en deux étapes}};

% The second level, relatively positioned nodes
\begin{scope}[every node/.style={level 3}]
\node [below of = c1, xshift=5pt, yshift=-11pt] (c11) {Co-consolidation};
\node [below of = c11, yshift=-6pt] (c12) {Adhésifs à chaud};
\node [below of = c12, yshift=-13pt] (c13) {Assemblage avec film \mbox{amorphe}};

\node [below of = c2, xshift=5pt, yshift=-11pt] (c21) {Soudage par friction};
\node [below of = c21, yshift=-6pt] (c22) {Soudage par vibrations};
\node [below of = c22, yshift=-6pt] (c23) {Soudage par ultrasons};

\node [below of = c3, xshift=5pt, yshift=-11pt] (c31) {Soudage par induction};
\node [below of = c31, yshift=-6pt] (c32) {Soudage par micro-ondes};
\node [below of = c32, yshift=-6pt] (c33) {Soudage \mbox{diélectrique}};
\node [below of = c33, yshift=-6pt] (c34) {Soudage \mbox{par résistance}};

\node [below of = c4, xshift=5pt, yshift=-17pt] (c41) {Soudage par plaque \mbox{chauffante}};
\node [below of = c41, yshift=-14pt] (c42) {Soudage au gaz chaud};
\node [below of = c42, yshift=-6pt] (c43) {Soudage \mbox{infrarouge}};
\node [below of = c43, yshift=-6pt] (c44) {Soudage au laser};
\end{scope}

% lines from each level 1 node to every one of its "children"
\foreach \value in {1,2,3}
	\draw[->] (c1.195) |- (c1\value.west);

\foreach \value in {1,...,3}
	\draw[->] (c2.195) |- (c2\value.west);

\foreach \value in {1,...,4}
	\draw[->] (c3.195) |- (c3\value.west);

\foreach \value in {1,...,4}
	\draw[->] (c4.195) |- (c4\value.west);
\end{tikzpicture}
