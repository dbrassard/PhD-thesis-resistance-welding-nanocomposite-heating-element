% Abstract
%
% Résumé de la recherche écrit en anglais sans être
% une traduction mot à mot du résumé écrit en français.

\chapter*{ABSTRACT}\thispagestyle{headings}
\addcontentsline{toc}{compteur}{ABSTRACT}
%
\begin{otherlanguage}{english}
	
	Resistance welding is a quick and efficient way to join thermoplastic composite parts. 
	In collaboration with the industrial partner, ArianeGroup, this project develops a nanocomposite heating element to produce a flexible junction by welding thermoplastic composite parts to a thermoplastic elastomer. 
	This welded flexible joint will replace an adhesive bonded junction between the tanks and skirts of a satellite launcher. 
	Development of this joint was done in 4 stages. 
	First, it was necessary to develop a conductive nanocomposite with appropriate properties to serve as a heating element during resistance welding. 
	This heating element then served to develop a resistance welding process between thermoplastic composite adherents. 
	Following that, a process window was established to improve our control over the process and the performance of the joints. 
	Finally, the nanocomposite heating element served to develop a resistance welding process with a thermoplastic elastomer to form a flexible joint. 
	
	To obtain the electrical properties required to serve as a heating element and after having evaluated many compositions, a nanocomposite composed of PEI and 10\% weight fraction of MWCNT was selected. 
	Due to the variability in the results obtained under constant voltage operation of the power source, a constant power control mode was developed and successfully applied to produce welded joints with the nanocomposite heating element. 
	
	By using the data acquired during welding experiments, a finite element model was developed to predict the temperature profile in the weld and to establish a window of operation. 
	With this widow of operation, supplementary welding tests were performed and a lap shear strength of \SI{24.9}{\mega\pascal} was reached, an improvement of 28\% over our previous results. 
	The temperatures reached within the joint during welding experiments are, however, higher than temperatures traditionally necessary to obtain a weld with a stainless steel heating element, leading to thermal degradation. 
	It is hypothesized that the high mass fraction of MWCNT limits the polymer chain's mobility and delay their diffusion. 
	Furthermore, the high mass fraction of MWCNT reduces the ductility and strength of the nanocomposite. 
	
	The last portion of this work concentrates on the development of a resistance welding process to produce flexible joints with a thermoplastic elastomer. 
	After unsuccessful attempts with SIS and SEBS elastomers, encouraging results were obtained during isothermal welding tests in a vacuum oven and a hot press with the ULTEM STM1500 elastomer. 
	Despite its good thermal stability at high temperature, welding tests with this elastomer did not produce joints meeting the performance targets of ArianeGroup. 
	Once again, the elevated temperature during the process combined with the reduced mobility of the polymer chains could be the causes of these negative results. 
	
	Finally, based on the results obtained, research topics to improve the resistance welding process with a nanocomposite heating element are proposed. 
	
\end{otherlanguage}
